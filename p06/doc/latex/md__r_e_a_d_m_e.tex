\subsection*{Objetivo}

Esta práctica consistirá en la realización de un programa escrito en C++ que lea desde un fichero las especificaciones de un autómata finito determinista (D\+FA) y, a continuación, simule el comportamiento del autómata para una cadena que se dé como entrada.

Fichero para la definición del autómata finito determinista El autómata finito determinista vendrá definido en un fichero de texto con extensión .dfa y que tendrá el siguiente formato\+:


\begin{DoxyItemize}
\item Línea 1\+: Número total de estados del D\+FA
\item Línea 2\+: Estado de arranque del D\+FA
\item A continuación vendrá una línea para cada uno de los estados. Cada línea contendrá los siguientes números, separados entre sí por espacios en blanco\+:
\begin{DoxyItemize}
\item Número identificador del estado. Los estados del autómata se representarán mediante números enteros sin signo. La numeración de los estados corresponderá a los primeros números naturales comenzando por 0.
\item Un 1 si se trata de un estado de aceptación y un 0 si se trata de un estado de no aceptación.
\item Número de transiciones que posee el estado.
\item A continuación, para cada una de las transiciones, y separados por espacios en blanco, se detallará la información siguiente\+:
\begin{DoxyItemize}
\item Símbolo de entrada necesario para que se produzca la transición.
\item Estado destino de la transición.
\end{DoxyItemize}
\end{DoxyItemize}
\end{DoxyItemize}

A modo de ejemplo, se muestra un autómata junto con la definición del mismo especificada mediante un fichero .dfa



El programa deberá detectar que no haya ningún error en la definición del autómata. Esto es, habría que analizar que se cumplen las siguientes condiciones\+:


\begin{DoxyItemize}
\item Existe uno y sólo un estado inicial para el autómata.
\item Para cada estado del autómata siempre existe una y sólo una transición para cada uno de los símbolos del alfabeto.
\end{DoxyItemize}

\subsection*{Funcionamiento general del programa}

El programa principal debería ofrecer al usuario las siguientes opciones\+:


\begin{DoxyItemize}
\item Leer D\+FA\+: al seleccionar esta opción se deberá solicitar al usuario que introduzca el nombre del fichero .dfa donde se encuentra la especificación del autómata. A continuación, se deberá crear el autómata a partir de la especificación dada en el fichero. Habrá que notificar al usuario si se produce algún error en la creación del D\+FA.
\item Mostrar D\+FA\+: al seleccionar esta opción se mostrará por pantalla el D\+FA actualmente definido (previamente leído de fichero) en nuestro programa. Para mostrar el D\+FA por pantalla se seguirá el formato establecido para los ficheros .dfa
\item Identificar estados de muerte\+: al seleccionar esta opción se deberá mostrar por pantalla si el autómata previamente definido tiene estados de muerte y si es así, habrá que indicar cuáles son los identificadores de dichos estados de muerte.
\item Analizar cadena\+: al seleccionar esta opción se deberá solicitar al usuario que introduzca una cadena. Para la cadena indicada por el usuario se deberá determinar si es aceptada o no por el autómata actualmente definido. Al igual que ocurre con las dos opciones anteriores, esta opción tampoco se podrá ejecutar hasta que se haya definido un D\+FA. El formato de la traza a mostrar por pantalla será el siguiente\+: \begin{DoxyVerb}      Cadena de entrada: ___________
      Estado actual    Entrada    Siguiente estado
      X                Y          Z
      X                Y          Z
      X                Y          Z
      Cadena de entrada ACEPTADA / NO ACEPTADA
\end{DoxyVerb}

\end{DoxyItemize}

Tal y como se indica en las líneas anteriores, en primer lugar se deberá mostrar la cadena de entrada. A continuación se indicará cómo va transitando de un estado a otro el autómata según va leyendo la cadena de entrada y finalmente, se deberá mostrar el mensaje “\+Cadena de entrada A\+C\+E\+P\+T\+A\+D\+A” si la cadena de entrada es aceptada por el D\+FA o el mensaje “\+Cadena de entrada NO A\+C\+E\+P\+T\+A\+D\+A” si la cadena de entrada no es aceptada por el D\+FA.

\subsection*{Detalles de implementación}


\begin{DoxyItemize}
\item Para la implementación de esta práctica se deberá definir una clase D\+FA que proporcione al menos los métodos necesarios para leer e inicializar un D\+FA desde un fichero, para mostrar un D\+FA, para identificar los estados de muerte del D\+FA y para realizar una traza de las transiciones llevadas a cabo cuando se da como entrada una determinada cadena.
\item También se deberá definir una clase \char`\"{}\+Estado\char`\"{} que agrupe las características correspondientes al estado de un D\+FA y las funcionalidades asociadas. Se valorará un buen diseño orientado a objetos basado en la definición formal de un D\+FA (\char`\"{}... un automata finito determinista D\+F\+A se define como un conjunto de estados ....\char`\"{} )
\item Se requerirá el uso de Makefiles para compilar los códigos.
\item Se requerirán los códigos de cada clase en archivos separados (cada clase con su .h y su .cpp) 
\end{DoxyItemize}